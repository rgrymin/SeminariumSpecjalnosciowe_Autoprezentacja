\input{preamble}


\begin{document}
% \usebackgroundtemplate{\includegraphics[width=\paperwidth]{img/bg.jpg}}
\frame{\titlepage}

% Przebieg prezentacji
\begin{frame}
  \frametitle{Przebieg prezentacji}
  \tableofcontents
\end{frame}

% The following causes the table of contents to be shown 
% at the beginning of every subsection. Delete this, if you do not want it.
\AtBeginSubsection[]{
  \frame<beamer>{
    \frametitle{Przebieg prezentacji}
    \tableofcontents[currentsection,currentsubsection]
  }
}

\section{Autoprezentacja}
\subsection{Projekt inżynierski}
\begin{frame}
  \frametitle{System ultradźwiękowy ze spłaszczoną wiązką emisyjną}
  
  \begin{itemize}
  	\item Projekt realizowany u dr Bogdana Kreczmera na specjalności Robotyka
  \end{itemize}
  \begin{figure}
    \includegraphics[width=0.2\textwidth]{img/zewnatrz_tyl}
    \hspace{1in}
    \includegraphics[width=0.2\textwidth]{img/zewnatrz_przod} 
    \bigskip
    
    \includegraphics[width=0.2\textwidth]{img/srodek_tyl} 
    \hspace{1in}
    \includegraphics[width=0.2\textwidth]{img/tyl} 
  \end{figure}
\end{frame}

\begin{frame}
  \frametitle{Zastosowanie}
  
  Wykrywanie ścian oraz narożników pokoju
  za pomocą pojedynczej emisji wiązki ultradźwiękowej.
\end{frame}

\subsection{Zainteresowania}
\begin{frame}
  \frametitle{Zainteresowania inżynieryjne}
  
  Do moich zainteresowań inżynieryjnych zaliczam: \pause
  \begin{itemize}
  	\item języki programowania wspierające paradygmat obiektowy, \pause
  	\item metaprogramowanie,                                     \pause
  	\item wzorce projektowe,                                     \pause
  	\item współbieżność,                                         \pause
  	\item systemy kontroli wersji (Git/SVN),                     \pause
  	\item metodyka zarządzania Lean Six Sigma.
  \end{itemize}
\end{frame}

\begin{frame}{Żeglarstwo}
	Dorobek żeglarski:	
	\begin{itemize}
		\item jachtowy sternik morski,                                    \pause
		\item sternik na dwóch koloniach żeglarskich,
		  instruktor żeglarstwa na dwóch obozach żeglarskich,             \pause
		\item udział w międzynarodowych regatach Tall Ships' Races 2009
		  jako starszy wachty na s/y Gedania,                             \pause
		\item 1 miejsce w kategorii C w międzynarodowych regatach
		  The Culture 2011 Tall Ships Regatta jako członek załogi,        \pause
		\item oraz wiele innych mniej ważnych obozów żeglarsko-harcerskich i rejsów...          
	\end{itemize}
\end{frame}
	
	\begin{frame}[allowframebreaks]{Żeglarstwo --- zdjęcia}
	\begin{figure}[!htp]
		\includegraphics[width=0.48\textwidth]{img/mazury} 
		\hspace{0.02\textwidth}
		\includegraphics[width=0.48\textwidth]{img/roza} 
		\caption{Sternik oraz instruktor żeglarski na Mazurach}
	\end{figure}

	
	\begin{figure}[!htp]
		\includegraphics[width=0.48\textwidth]{img/tsr2}
		\hspace{0.02\textwidth}
		\includegraphics[width=0.48\textwidth]{img/tsr3}
		\caption{Regaty The Culture 2011 Tall Ships Regatta}
	\end{figure}
	
	\begin{figure}[!htp]
		\includegraphics[width=0.48\textwidth]{img/kruzernstern}
		\hspace{0.02\textwidth}
		\includegraphics[width=0.48\textwidth]{img/flaga}
		\caption{Regaty The Culture 2011 Tall Ships Regatta}
	\end{figure}
	
	\begin{figure}[!htp]
		\includegraphics[width=0.48\textwidth]{img/chorwa}
		\hspace{0.02\textwidth}
		\includegraphics[width=0.48\textwidth]{img/chorwa2}
		\caption{Sylwester 2009 w Chorwacji}
	\end{figure}
	
	\begin{figure}[!htp]
		\includegraphics[width=0.48\textwidth]{img/chorwa3}
		\hspace{0.02\textwidth}
		\includegraphics[width=0.48\textwidth]{img/chorwa4}
		\caption{Sylwester 2009 w Chorwacji}
	\end{figure}
\end{frame}

\subsection{Praca}


\begin{frame}{Nokia Siemens Networks}
  \framesubtitle{(niedługo nastąpi zmiana nazwy na Nokia Solutions and Networks)}
  Jestem pracownikiem Nokia Siemens Networks:
  \begin{itemize}
  	\item stanowisko: inżynier ds. rozwoju oprogramowania,  \pause
  	\item dział: WCDMA (ang. \textit{Wideband Code Division Multiple Access}), \pause
  	\item domena: ALMAG (ang. \textit{Antena Line Management}).
  \end{itemize}
  
  \begin{figure}[!htp]
  	\includegraphics[width=0.45\textwidth]{img/gtb}
  	\hspace{0.38\textwidth}
  	\includegraphics[width=0.15\textwidth]{img/logo}
  \end{figure}
\end{frame}

\section{Plan seminarium}
\subsection{Wybór tematu}
\begin{frame}{Inteligentne wskaźniki}
	Na seminarium specjalnościowym poruszę temat inteligentnych wskaźników.
	Odpowiem m.in. na następujące pytania: \pause
	\begin{itemize}
		\item Czym są inteligentne wskaźniki? Jakie korzyści niosą ze sobą? \pause
		\item Jakie języki posiadają wbudowane inteligentne wskaźniki?       \pause
		\item Czym są RAII i \texttt{std::auto\_ptr} (często mylnie nazywany jest
		  inteligentnym wskaźnikiem) i dlaczego \texttt{std::auto\_ptr}
		  wychodzi z użycia?              \pause
		\item Jakie inteligentne wskaźniki udostępnia biblioteka Boost?  \pause
		  Jak ich używać?
	\end{itemize}
\end{frame}



\section{}
\begin{frame}
  \begin{center}
    \huge
    Dziękuję za uwagę!
  \end{center}
  \bigskip
  
      \begin{figure}
      \includegraphics[scale=0.5]{img/death-by-beamer}
    \end{figure}
 

\end{frame}

\end{document}
